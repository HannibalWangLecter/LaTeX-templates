

\chapter{Figures, tables and images} \label{chap-3}

\section{Figures}

\begin{figure}
\center
\includegraphics[width=0.3\textwidth]{chap3/emblem.jpg} 
\caption[Short caption for List of Figures]{{\bfseries Short caption (if wanted).} Full caption with all the details here.}
\label{fig-example}
\end{figure}

\begin{figure}
\center
\includegraphics[width=0.3\textwidth]{chap3/symbol.jpg} 
\caption*{This secret image won't be numbered and won't appear in the List of Figures because of the *}
\end{figure}


Figures should appear as close as possible to the first mention of them in the text. All figures must be referred to in the text by either a parenthetical mark-up (Figure~\ref{fig-example}) or a phrasing such as ``Sequencing data, shown in Figure~\ref{fig-example}, shows that...''.  A parenthetical mention, but not an in-text mention, may be abbreviated as (Fig.~\ref{fig-example}).  The number of the chapter should be part of the Figure number.

Figures must be accompanied by a caption that describes the material clearly and succinctly. Figure captions may start with a brief title in bold, which can then be referenced in the list of figures. 

As a general rule, figures should not have captions that run across pages.  If a figure and its caption will be larger than one page, rewriting should be considered, or a reorganization of the figure.  If this cannot be avoided, the figure caption should continue on the immediate next page, with a reference comment at the start of the text to the fact that it is a continuation.  No other main body text should then appear on that page.

\section{Tables}

\begin{table} 
\center
\caption{Short heading for the List of Tables.}
\begin{tabular}{c|c}
Parameter & Value \\ \hline \hline
$\Delta$ & 0, 150 \\
${\alpha}$ & 85 \\
${\epsilon}$ & 6 \\
${\kappa}$ & 6.8 \\
${\gamma}$ & 0.2
\end{tabular}
\label{tab-values}
\caption*{Full caption with all the details here.}
\end{table}

\begin{table} \center
\begin{tabular}{c|c}
Parameter & Value \\ \hline \hline
$\Delta$ & 0, 1500 \\
${\alpha}$ & 850 \\
${\epsilon}$ & 60 \\
${\kappa}$ & 68 \\
${\gamma}$ & 2
\end{tabular}
\caption*{This secret table won't be numbered and won't appear in the List of Figures because of the * }
\end{table}

All tables should be referred to in the text by number (for example) ``Table ~\ref{tab-values} describes all particles found in...''.  Tables may be printed in landscape mode rather than portrait mode, but must then be printed on a separate page (with continuing and sequential pagination). Tables may extend for more than one page, but should then have the table header row repeated on each page. Do not use font sizes smaller than 9 point. Tables should have a heading and may have a caption.  The number of the chapter should be part of the Table number.


\section{Images}

Images are vital to presentation of scientific data.  Textual annotations must be correctly labelled, and legends, when used, must be clear and legible.  Small symbols should be used on charts for data points.  Axis marks and axis labels should be large enough to be read clearly.  All white space should be used where possible.  Headings for charts and captions explain the data within should be meaningful.  Students must be aware of expected standards covering image manipulation and the standard practice for image presentation within their field and adhere to it.  Excessive density, contrast, and hue manipulation of photographic images should be avoided.  Where extensive manipulation of images is required for data extraction or analysis, this must be clearly explained in the description of methodology, and explicitly in the caption for each figure.